\documentclass{llncs}

\usepackage[margin=1in]{geometry}

\usepackage{graphicx,color,comment,url} 

\usepackage{amsmath,amssymb}


\title{CS4033/5033: Assignment 5}
\author{Your Name}
\institute{}

\begin{document}

\maketitle 

\setlength\parindent{0pt} 
\setlength{\parskip}{10pt}

\textbf{Topic 1: Low-Rank Matrix Factorization}

Implement low-rank matrix factorization 
based on the alternate least square technique. 
Evaluate the technique on a real-world user-movie 
rating matrix. In this matrix, each row corresponds to a user and each column corresponds to a movie. Ratings take value in 
\{1, 1.5, 2, 2.5, 3, 3.5, 4, 4.5, 5\}; you should 
treat `0' entries in the matrix as (truly) missing ratings 
and do not use it for either training or testing. 
For convenience, we have separated the set of 
observed ratings that should be used for training 
(stored in `rate\_train.csv') from the set of 
observed ratings that should be used for testing 
(stored in `rate\_test.csv'). 

\underline{Task 1}. 
Learn a rank-$k$ factorization 
of the rating matrix from the training set and 
evaluate it on the testing set. Report testing error 
versus the number of ALS updates in Figure \ref{fig1}. 
Pick the value for $k$, $\lambda_1$, $\lambda_2$ and the maximum number of ALS updates by yourself. 

Tip: you may round the predicted rates 
(given by $UV$) to the nearest value in 
\{1, 1.5, 2, 2.5, 3, 3.5, 4, 4.5, 5\} 
to get a more meaningful prediction 
and possibly higher prediction accuracy. 

\underline{Task 2}. Report testing error versus $k$ in Figure \ref{fig2}. Pick five values of $k$ and 
fix other configurations by yourself. 


\begin{figure}[]
\centering
\includegraphics[width=.8\textwidth]{ }
\caption{Testing Error versus ALS Updates with $k = \ldots$}
\label{fig1}
\end{figure}

\begin{figure}[]
\centering
\includegraphics[width=.8\textwidth]{ }
\caption{Testing Error versus $k$}
\label{fig2}
\end{figure}

\textbf{Topic 2: Markov Model}

Suppose we want to learn a Markov model, 
and apply it to predict Sam's final grade 
as A or B. Our training data are shown 
in Figure \ref{fig3}. (The first row is 
Sam's record. Do not use it for training.) 

\underline{Task 3}. Assume 1st-order Markov 
chain. Evaluate the probability for Sam to 
get an A in the 5th semester. You need to 
elaborate on the estimation of key probabilities 
and the key formula such as Bayes theorem. 


\underline{Task 4}. Assume 2nd-order Markov 
chain. Evaluate the probability for Sam to 
get an A in the 5th semester. You need to 
elaborate on the estimation of key probabilities 
and the key formula such as Bayes theorem. 

\begin{figure}[]
\centering
\includegraphics[width=.7\textwidth]{figure/markov.PNG}
\caption{}
\label{fig3}
\end{figure}

\vfill 

\underline{Submission Instruction}

Please submit three files on Canvas. 

(i) Submit a `hw5.pdf'. It should contain 
your answers to all questions in this template. 

(ii) Submit a `hw5\_als.py'. It should 
be the code that generates Figure \ref{fig1}.

(iii) Submit a `hw5\_k.py'. It should 
be the code that generates Figure \ref{fig2}.

\end{document}
